\documentclass{article}
\usepackage[margin=1in]{geometry}
\usepackage{graphicx} % Required for inserting images

\usepackage{amsmath, amssymb}
\usepackage{diffcoeff, cancel}

\title{Cahn-Hilliard with Deal.II}
\author{Aiden Huffman}
\date{October 2023}

\begin{document}

\maketitle

For our purposes, the Cahn-Hilliard equation is described by the following
partial differential equation
\begin{align}
    \diffp{\phi}{t} &= \nabla\cdot(M(\phi)\nabla\eta)\\
    \eta(\phi) &= F'(\phi)-\epsilon^2 \nabla^2 \phi\\
\end{align}
where
\begin{equation}
    F(\phi) = \frac{1}{4}(\phi^2-1)^2,\qquad M(\phi) = 1-\phi^2
\end{equation}
so that
\begin{align}
    \diffp{\phi}{t} &= \nabla\cdot((1-\phi^2)\nabla\eta)\\
    \eta(\phi) &= \phi(\phi^2-1) - \epsilon^2\nabla^2\phi 
\end{align}
We'll begin by solving this in the simplest way possible. In particular, a first
order forward Euler method. Before this, we will replace the $\phi$'s in the
system with $c$ since we frequently use $\phi$ to represent functions in the
trial space. As a consequence, our system takes the form:
\begin{align}
\frac{c^{n+1}-c^{n}}{\Delta t} &= \nabla\cdot((1-(c^n)^2)\nabla\eta^n))\\
\eta^n &= c^n((c^n)^2-1) - \epsilon^2 \nabla^2 c^n
\end{align}
Notice that we can solve for $\eta^n$ before finding $c^{n+1}$ and provides a
reasonable method for updating the system. Projecting into the test space, we
have
\begin{align}
    (\phi, c^{n+1})_\Omega  &= (\phi, c^n)_\Omega
                            + \Delta t (
                                \phi,\,
                                \nabla\cdot(1-(c^n)^2)\nabla(\eta^n)
                            )_\Omega,\\
    (\phi, \eta^n)_\Omega   &= (\phi, c((c^n)^2-1)))_\Omega 
                            - \epsilon^2(\phi,\nabla^2 c^n).
\end{align}
After integrating by parts, and assuming periodic boundary conditions
\begin{align}
    (\phi, c^{n+1})_\Omega  &= (\phi, c^n)_\Omega
                            - \Delta t (
                                \nabla \phi,\,
                                (1-(c^n)^2)\nabla(\eta^n)
                            )_\Omega,\\
    (\phi, \eta^n)_\Omega   &= (\phi, c((c^n)^2-1)))_\Omega 
                            + \epsilon^2(\nabla \phi,\nabla c^n).
\end{align}
If we assume that the solution lives in the trial space, and take some finite
basis of the function space we can replace $(\phi, c^{n})$ with a mass matrix,
$(\nabla \phi, \nabla c^n)$ with a laplace matrix and perform similar
substitutions for $\eta$. If $M$ and $A$ represent the two matrices, the system
is approximately governed by
\begin{align}
    MC^{n+1} &= MC^n - \Delta t (\nabla \phi_i, F(c^n, \eta^n))\\
    M\eta^n &= \epsilon^2 A C^n + (\phi, G(c^n))
\end{align}
\end{document}
