\documentclass{article}
\usepackage[utf8]{inputenc}
\usepackage[margin=1in]{geometry}
\usepackage{graphicx, biblatex} % Required for inserting images

\usepackage{amsmath, amssymb}
\usepackage{diffcoeff, cancel}

\title{Cahn-Hilliard with Constant Mobility}
\author{Aiden Huffman}
\date{November 2023}

\newcommand\ang[1]{\left\langle #1 \right\rangle}

\begin{document}

\maketitle

\subsection*{Direct Solve}

We consider the Cahn-Hilliard equation of the following form
\begin{align}
    \diffp{\phi}{t} &= \nabla^2\eta(\phi)\\
    \eta(\phi) &= f'(\phi) - \epsilon^2\nabla^2 \phi
\end{align}
where $f(\phi) = (\phi^2-1)^2/4$. This can be reorganized as
\begin{align}
    \diffp{\phi}{t} &= \nabla^2 \left(
        \phi^3-\phi - \epsilon^2\nabla^2 \phi
    \right)\\
                    &= \nabla^2 \phi^3  - \nabla^2 \phi - \epsilon\nabla^4 \phi
\end{align}
We will consider a backward Euler method, which can be easily modified into a
Crank-Nicoholson scheme later.
\begin{equation}
    \phi^n - \phi^{n-1} - \Delta t\left(
        \nabla^2 (\phi^n)^3  - \nabla^2 \phi^n - \epsilon^2\nabla^4 \phi^n
    \right) = 0
\end{equation}
Multiplying on the left by a test function $\varphi$
\begin{equation}
    \ang{\varphi,\phi^n}_\Omega - \ang{\varphi, \phi^{n-1}}_{\Omega}
    - \Delta t \left(
        \ang{\varphi, \nabla^2(\phi^n)^3}_\Omega 
        - \ang{\varphi, \nabla^2 \phi}_\Omega
        - \epsilon^2\ang{\varphi,\nabla^4\phi}_\Omega
    \right) = 0
\end{equation}
Integrating the more troublesome equations by parts
\begin{equation}
    \ang{\varphi,\phi^n}_\Omega - \ang{\varphi,\phi^{n-1}}_\Omega
    + \Delta t \left(
        \ang{\nabla\varphi, 3(\phi^n)^2\nabla\phi^n}_\Omega
        - \ang{\nabla\varphi, \nabla\phi^n}_\Omega
        + \ang{\nabla^2\varphi, \nabla^2\phi^n}_\Omega
    \right) = 0 
\end{equation}
Assume that $\phi$ and $\varphi$ live in the same finite dimensional function
space which is spanned by $\varphi_i$, then
\begin{equation}
    \ang{\varphi_i, \varphi_j}_\Omega + \Delta t\left(
        \ang{\nabla^2 \varphi_i,\nabla^2\varphi_j}_\Omega
        - \ang{\nabla \varphi_i,\nabla\varphi_j}_\Omega
    \right) = \ang{\varphi_i, \varphi_j}\phi^{n-1}_j
    - \Delta t\ang{\varphi_i, 3(\phi^{n-1})^2\nabla \phi^n}_\Omega
\end{equation}
Constructing the finite element space for this will require second order
elements and is particularly expensive. We are better off trying to determine a
way of implementing by introducing a dummy variable

\subsection*{Newton with a Dummy}

\begin{align}
    \diffp{\phi}{t} &= \nabla^2 \phi^3
                    - \nabla^2 \phi
                    - \epsilon^2\nabla^2 \eta,\\
    \eta            &= \nabla^2 \phi
\end{align}
We quickly compute the temporal discretization of the problem, substituting
$k$ for $\Delta t$ which is less cumbersome,
\begin{align}
    &\phi^n - k(\nabla^2 (\phi^n)^3 - \nabla^2 \phi - \epsilon^2\nabla^2 \eta^n) 
        = \phi^{n-1},\\
    &\eta^n = \nabla^2 \phi^n.
\end{align}
projecting into the test space gives
\begin{align}
    &\ang{u, \phi^n}_\Omega -k\left(
        \ang{-\nabla u, 3(\phi^n)^2\nabla\phi^n}_\Omega
        + \ang{\nabla u, \nabla \phi^n}
        + \epsilon^2\ang{\nabla u,\nabla \eta^n}
    \right) = \ang{u, \phi^{n-1}}_\Omega\\
    &\ang{u, \eta^n}_\Omega = \ang{\nabla u, \nabla \phi^n}_\Omega
\end{align}
To derive the Newton step let $\phi^n = \phi^{n,k} + \delta \phi^n$ then
\begin{align}
    &\ang{u, \phi^{n,k}}_\Omega + \ang{u,\delta\phi^n}_\Omega 
        - k \ang{\nabla u, 3(\phi^{n,k}+\delta\phi^n)^2\nabla(\phi^{n,k}+\delta\phi^n)}_\Omega\\
    &\hspace{4ex}+ k\ang{\nabla u, \nabla (\phi^{n,k} + \delta\phi^n} + k\epsilon^2\ang{\nabla u,\nabla \eta(\phi^{n,k} + \delta\phi^n)} = \ang{u, \phi^{n-1}}_\Omega
\end{align}
If we assume that $(\delta \phi^n)^2 \approx 0$ and
$\delta\phi^n\nabla(\delta\phi^n)\approx 0$, we obtain
\begin{align}
    &\ang{u, \phi^{n,k}}_\Omega + \ang{u,\delta\phi^n}_\Omega\\
    &\hspace{4ex}- k \ang{\nabla u, 3(\phi^{n,k})^2\nabla(\phi^{n,k})}\\
    &\hspace{4ex}- k \ang{\nabla u, 3(\phi^{n,k})^2\nabla(\delta \phi^n)}_\Omega\\ 
    &\hspace{4ex}- k \ang{\nabla u, 6(\delta\phi^n)\phi^{n,k}\nabla\phi^{n,k}}_\Omega\\
    &\hspace{4ex}+ k\ang{\nabla u, \nabla \phi^{n,k}}_\Omega\\ 
    &\hspace{4ex}+ k\ang{\nabla u, \nabla\delta\phi^n}_\Omega\\ 
    &\hspace{4ex}+ k\epsilon^2\ang{\nabla u,\nabla \eta(\phi^{n,k})}_\Omega\\
    &\hspace{4ex}+ k\epsilon^2\ang{\nabla u, \nabla\eta(\phi^{n,k})}\\
    &\hspace{4ex}+ k\epsilon^2\ang{\nabla u, \nabla\eta'(\phi^{n,k})\delta\phi^n)}_\Omega\\
    &\hspace{4ex}= \ang{u, \phi^{n-1}}_\Omega
\end{align}
Constructing the linear system for $\delta\phi^n$ we find that
\begin{align}
    &\ang{\varphi_i, \varphi_j}\delta\phi^n_j\\
    &\hspace{4ex}- k\ang{\nabla \varphi_i, 3(\phi^{n,k})^2\nabla\varphi_j}\delta\phi^n_j\\
    &\hspace{4ex}- k\ang{\nabla \varphi_i, (6\phi^{n,k}\nabla\phi^{n,k})\varphi_j}\delta\phi^n_j\\
    &\hspace{4ex}+ k\ang{\nabla \varphi_i, \nabla\varphi_j}\delta\phi^n_j\\
    &\hspace{4ex}+ k\epsilon^2\ang{\nabla\varphi_i, \nabla\eta'(\phi^{n,k})\varphi_j}\delta\phi^n_j\\
    &= \ang{\varphi_i, \phi^{n-1}}\\
    &\hspace{4ex}-\ang{\varphi_i, \varphi_j}\phi^{n,k}_j\\
    &\hspace{4ex}+k\ang{\nabla\varphi_i, 3(\phi^{n,k})^2\nabla\phi^{n,k}}\\
    &\hspace{4ex}-k\ang{\nabla\varphi_i, \nabla\varphi_j}\phi^{n,k}_j\\
    &\hspace{4ex}-k\epsilon^2\ang{\nabla\varphi_i,\eta(\phi^{n,ki})}
\end{align}
By inverting the linear system we can find $\delta\phi^n$ and define our update
$\phi^{n,k+1} = \phi^{n,k} + \alpha\delta \phi^n$; however, this raises an obvious
question, what the heck is $\eta'(\phi^{n,k})$? Being entirely unrigourous,
in part because the last time I saw this sort of functional analysis was several
years ago we have something like this
\begin{align}
    \eta(\phi+\delta \phi) &= \nabla^2\phi + \nabla^2 \delta \phi\\
    &= \eta(\phi) + \eta'(\phi)\delta\phi
\end{align}
The take away being that $\nabla^2$ is a linear operator, so it's linearization
is itself.
\begin{align}
    &\ang{\varphi_i, \varphi_j}\delta\phi^n_j\\
    &\hspace{4ex}- k\ang{\nabla \varphi_i, 3(\phi^{n,k})^2\nabla\varphi_j}\delta\phi^n_j\\
    &\hspace{4ex}- k\ang{\nabla \varphi_i, (6\phi^{n,k}\nabla\phi^{n,k})\varphi_j}\delta\phi^n_j\\
    &\hspace{4ex}+ k\ang{\nabla \varphi_i, \nabla\varphi_j}\delta\phi^n_j\\
    &\hspace{4ex}+ k\epsilon^2\ang{\nabla\varphi_i, \nabla^3\varphi_j}\delta\phi^n_j\\
    &= \ang{\varphi_i, \phi^{n-1}}\\
    &\hspace{4ex}-\ang{\varphi_i, \varphi_j}\phi^{n,k}_j\\
    &\hspace{4ex}+k\ang{\nabla\varphi_i, 3(\phi^{n,k})^2\nabla\phi^{n,k}}\\
    &\hspace{4ex}-k\ang{\nabla\varphi_i, \nabla\varphi_j}\phi^{n,k}_j\\
    &\hspace{4ex}-k\epsilon^2\ang{\nabla\varphi_i,\eta(\phi^{n,ki})}
\end{align}
integrating by parts we are actually left in the same situation as before
\begin{align}
    &\ang{\varphi_i, \varphi_j}\delta\phi^n_j\\
    &\hspace{4ex}- k\ang{\nabla \varphi_i, 3(\phi^{n,k})^2\nabla\varphi_j}\delta\phi^n_j\\
    &\hspace{4ex}- k\ang{\nabla \varphi_i, (6\phi^{n,k}\nabla\phi^{n,k})\varphi_j}\delta\phi^n_j\\
    &\hspace{4ex}+ k\ang{\nabla \varphi_i, \nabla\varphi_j}\delta\phi^n_j\\
    &\hspace{4ex}- k\epsilon^2\ang{\nabla^2\varphi_i, \nabla^2\varphi_j}\delta\phi^n_j\\
    &= \ang{\varphi_i, \phi^{n-1}}\\
    &\hspace{4ex}-\ang{\varphi_i, \varphi_j}\phi^{n,k}_j\\
    &\hspace{4ex}+k\ang{\nabla\varphi_i, 3(\phi^{n,k})^2\nabla\phi^{n,k}}\\
    &\hspace{4ex}-k\ang{\nabla\varphi_i, \nabla\varphi_j}\phi^{n,k}_j\\
    &\hspace{4ex}-k\epsilon^2\ang{\nabla\varphi_i,\eta(\phi^{n,ki})}
\end{align}
\end{document}
