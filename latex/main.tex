\documentclass{article}
\usepackage[margin=1in]{geometry}
\usepackage{graphicx} % Required for inserting images

\usepackage{amsmath, amssymb}
\usepackage{diffcoeff, cancel}

\title{Cahn-Hilliard with Deal.II}
\author{Aiden Huffman}
\date{October 2023}

\begin{document}

\maketitle

For our purposes, the Cahn-Hilliard equation is described by the following
partial differential equation
\begin{align}
    \diffp{\phi}{t} &= \nabla\cdot((1-\phi^2)\nabla\eta)\\
    \eta(\phi) &= F'(\phi)-\epsilon^2 \nabla^2 \phi\\
\end{align}
where
\begin{equation}
    F(\phi) = \frac{1}{4}(1-\phi^2)^2
\end{equation}
so that
\begin{equation}
    F'(\phi) = -\phi(1-\phi^2)
\end{equation}
We'll begin by solving this in the simplest way possible. In particular, a first
order forward Euler method.
\begin{align}
\frac{c^{n+1}-c^{n}}{\Delta t} &= \nabla\cdot((1-(c^n)^2)\nabla\eta^n))\\
\eta^n &= -c^n(1-(c^n)^2) - \epsilon^2 \nabla^2 c^n
\end{align}
Multiplying through by the test function, we have
\begin{align}
    (\phi, c^{n+1})_\Omega &= (\phi, c^n)_\Omega
                            + \Delta t (\nabla \phi, (1-(c^n)^2)\nabla(\eta^n))_\Omega\\
    (\phi, \eta^n)_\Omega &= -(\phi, c^n(1-(c^n)^2))_\Omega 
                            - \epsilon^2 (\nabla\phi, \nabla c^n)_\Omega
\end{align}
We will impose periodic conditions on both boundaries. As a consequence, there
are no boundaries to integrate over. Since we are time stepping explicitly, we
can solve for $\eta^n$ independent of $c^{n+1}$ and use the previous solution
to construct the necessary values, after some work, we have:
\begin{align}
    MC^{n+1} &= MC^n + \Delta t (\nabla \phi_i, F(c^n, \eta^n))\\
    M\eta^n &= -\epsilon^2 A C^n - (\phi, G(c^n))
\end{align}
We could implicitly step the laplace term, however because $\epsilon^2$ is
small, this should generally not be super stiff. It would also couple the system
which may make it more difficult to solve for the next timestep.
\end{document}
